\chapter*{Введение}
\addcontentsline{toc}{chapter}{Введение}

Современное компьютерное моделирование и визуализация требуют детального и реалистичного представления трехмерных объектов. Особенное внимание уделяется методам, позволяющим достичь высокой степени реалистичности без заметного увеличения вычислительной сложности или объема данных. Одним из таких методов является учет текстуры на поверхности тел путем внесения возмущений в нормаль.

Текстурирование позволяет не только изменять внешний вид объекта, но и модифицировать его геометрические и световые характеристики. С помощью текстур можно создавать реалистичные детали поверхности, такие как мельчайшие неровности, шероховатости и даже более сложные элементы, такие как: камни, узоры и рельефы, не добавляя при этом дополнительных геометрических объектов в сцену.

Цель работы -- разработать программное обеспечения для учета текстуры на поверхности трехмерных тел с использованием метода внесения возмущения в нормаль.

Для достижения поставленной цели требуется решить следующие задачи:
\begin{enumerate}[label=\arabic*)]
    \item Определить, какие объекты будут располагаться на сцене;
    \item Выбрать алгоритмы для построения и обработки трехмерных объектов. Это включает в себя методы удаления невидимых поверхностей, создания теней и отражений;
    \item Определить модель освещения, которая будет использоваться для создания реалистичных световых эффектов;
    \item Спроектировать структуру программного обеспечения и выбрать подходящий способ представления данных;
    \item Выбрать средства реализации алгоритмов;
    \item Создать программное обеспечениe, реализовав в нем выбранные алгоритмы;
    \item Исследовать различные способы текстурирования, а также их временные характеристики на основе созданного программного обеспечения.
\end{enumerate}
