\chapter{Аналитическая часть}
В данном разделе производится формализация объектов сцены, анализ алгоритмов их визуализации, и выбирается наиболее подходящие для решения поставленной задачи.

\section{Текстура, как характеристика поверхности трехмерного тела}

В компьютерной графике текстурой называется детализация структуры поверхности. Текстура – это одномерное или двумерное изображение, которое имеет множество ассоциированных с ним параметров, определяющих, каким образом производится наложение изображения на поверхность.

Обычно рассматривается два вида детализации. Первый состоит в том, чтобы на гладкую поверхность нанести заранее заданный узор – так называемая детализация светом. После этого поверхность все равно остается гладкой. Второй тип детализации заключается в создании неровностей на поверхности, реализуется путем внесения возмущений в параметры поверхности – детализация фактурой. [1]

\section{Описание метода внесения возмущений в нормаль}

В этом методе для создания неровностей отдельно рассматривается каждый пиксел текстуры – текстел. Реалистичность изображения достигается за счет создания рельефа, а каждый текстел должен хранит информацию о возмущении в нормаль, которое требуется внести.

\section{Текстурные карты}

Для хранения информации о текстелях используют так называемые текстурные карты. Для метода внесения возмущений в нормаль чаще всего используют следующие:
\begin{itemize}[label=\arabic*)]
	\item[-] Карта высот (англ. height map). Каждый пиксел карты хранит одно значение. Поэтому изображения являются черно-белым, где белый – самые выпуклые точки на поверхности;
	\item[-] Карта нормалей (англ. normal map). Представлены в виде RGB-изображений, где каналы R и G описывают наклон каждой нормали;
	\item[-] Карта параллактического отображения (англ. parallax map). Эта карта использует особый визуальный эффект, чтобы создать иллюзию глубины и объемности. Она изменяет положение текстурных координат в зависимости от уровня смещения, создавая ощущение сложной геометрии на плоской поверхности.
\end{itemize}

\section{Представление объектов}

Для решения данной задачи будем использовать сферы. Такой простой тип объектов позволит хорошо рассмотреть наложение текстур на поверхность, а так же позволит сильно сократить время работы программы, что очень важно, поскольку мы хотим достичь максимальной детализации используя вычислительные возможности только процессора. Это дает нам возможность выбрать более затратные по времени алгоритмы, отдав приоритет визуальной составляющей результата.

\section{Анализ и выбор алгоритма удаления невидимых ребер и поверхностей}

Выделим самый важный критерий, которым должен обладать выбранный алгоритм:
\begin{itemize}[label=\arabic*)]
	\item[-] Получение высокореалистичных изображений. Для успешного наложения текстур важно учитывать даже небольшие изменения в освещении, тенях и других световых эффектах.
\end{itemize}

\subsection{Алгоритм буфера глубины}

Одним из самых распространенных алгоритмов удаления невидимых поверхностей является алгоритм буфера глубины (англ. Z-буфера). Здесь буфер глубины представляет собой дополнительное пространство в памяти, где для каждого пиксела хранится значение глубины объектов перед ним (расстояние от наблюдателя до поверхности объекта). Задача алгоритма заключается в нахождении минимального значения глубины $z$ для каждого пиксела экрана. [2, 3]

Принцип работы заключается в следующем: сначала весь буфер глубины
заполняется значениями, соответствующими максимальной глубине. В процессе растеризации граней для каждого пиксела рассчитывается его глубина и сравнивается с текущим значением в буфере глубины. Если рассматриваемый пиксел находится ближе, он рисуется, и его глубина заменяет значение в буфере. Если пиксел дальше, он не рисуется, и буфер остается неизменным – это позволяет отбросить невидимые поверхности.

Преимущества алгоритма:
\begin{itemize}[label=\arabic*)]
	\item[-] Простота реализации;
	\item[-] Высокая скорость работы;
	\item[-] Применим к различным типам объектов.
\end{itemize}

Недостатки:
\begin{itemize}[label=\arabic*)]
	\item[-] Производительность может снижаться при обработке сложных сцен;
	\item[-] Могут возникнуть проблемы с отображением прозрачных объектов;
	\item[-] Требуется значительный объем памяти.
\end{itemize}

\subsection{Алгоритм Робертса}

Алгоритм Робертса представляет собой первое известное решение задачи об удалении невидимых линий. Это математический метод, работающий в объектном пространстве. [2]

В алгоритме Робертса требуется, чтобы все изображаемые тела или объекты были выпуклыми. Невыпуклые тела должны быть разбиты на выпуклые части. Выпуклое многогранное тело с плоскими гранями должно представляться набором пересекающихся плоскостей.

Алгоритм Робертса включает в себя три этапа:
\begin{itemize}[label=\arabic*)]
	\item[-] На первом этапе каждое тело анализируется индивидуально с целью удаления нелицевых плоскостей;
	\item[-] На втором этапе проверяется экранирование оставшихся в каждом теле ребер всеми другими телами с целью обнаружения невидимых отрезков;
	\item[-] На третьем этапе вычисляются отрезки, которые образуют новые ребра при протыкании телами друг друга.
\end{itemize}

Преимущества алгоритма:
\begin{itemize}[label=\arabic*)]
	\item[-] Высокая точность изображения на выходе.
\end{itemize}
Недостатки:
\begin{itemize}[label=\arabic*)]
	\item[-] Сложность реализации;
	\item[-] Высокий рост сложности алгоритма при увеличении числа объектов;
	\item[-] Необходимость разбивать невыпуклые объекты на выпуклые, что может значительно замедлить выполнение программы.
\end{itemize}

\subsection{Алгоритм обратной трассировки лучей}

Методы трассировки лучей считаются наиболее мощными и универсальными методами создания реалистичных изображений. Существует множество успешных реализаций алгоритмов трассировки для отображения даже самых сложных трехмерных сцен. [2, 3]

Суть алгоритма заключается в том, что из источника наблюдения проводится луч в каждую точку картинной плоскости. Анализируя траекторию луча, мы можем определить, какие объекты он пересекает и от каких отражается. Это подобно тому, как человеческий глаз воспринимает световые лучи, но в обратном порядке.

При практической реализации метода обратной трассировки вводят ограничения. Вот некоторые из них:
\begin{enumerate}[label=\arabic*)]
	\item Среди типов объектов выделяются источники света, они могут только излучать свет;
	\item Свойства отражающих поверхностей описываются суммой двух компонент — диффузной и зеркальной;
	\item Зеркальность разбивается на две составляющие: отражение от других объектов (кроме источников света) и световые блики от источников;
	\item При диффузном отражении учитываются только лучи от источников света. Точки, находящиеся в тени, определяются тем, если луч на источник света блокируется другим объектом;
	\item Для прозрачных объектов часто упрощается моделирование преломления: без учета зависимости от длины волны. Прозрачность может также моделироваться без преломления, когда направление преломленного луча совпадает с направлением падающего луча.
\end{enumerate}

Преимущества алгоритма:
\begin{itemize}[label=\arabic*)]
	\item[-] Высокая степень реалистичности полученных изображений;
	\item[-] Вычислительная сложность не сильно зависит от количества объектов на сцене.
\end{itemize}
Недостатки:
\begin{itemize}[label=\arabic*)]
	\item[-] Ограничения в производительности из-за интенсивного вычислительного процесса.
\end{itemize}

Алгоритм обратной трассировки лучей является предпочтительным для использования в задачах внесения возмущений в нормали и текстурной модификации по следующим причинам:
\begin{itemize}[label=\arabic*)]
	\item[-] Алгоритм обратной трассировки лучей уже включает в себя вычисление нормалей для каждой точки поверхности. Это означает, что мы можем легко внести дополнительные изменения в нормали, соответствующие текстуре;
	\item[-] Алгоритм обратной трассировки лучей позволяет точно моделировать освещение, что имеет ключевое значение при работе с текстурными изменениями. Этот метод обеспечивает более реалистичные эффекты теней, подсветки и отражений, учитывая изменения, внесенные в нормали;
	\item[-] В данной задаче акцент делается на визуальное воздействие, а не на обработку больших объемов данных в кратчайшие сроки. Хотя алгоритм обратной трассировки лучей может потреблять больше ресурсов, он предоставляет значительно более качественное визуальное представление сцены.
\end{itemize}

\section{Выбор модели освещения}

При текстурировании в трассировке лучей, предпочтительной моделью
освещения является модель Фонга. Эта модель выделяется своей способностью учесть разнообразные характеристики поверхностей в зависимости от направления света и точки наблюдения. Такой выбор модели обеспечивает создание более реалистичных текстур на объектах.

Модель освещения Фонга включает как диффузную, так и зеркальную компоненты освещения. Диффузная составляющая позволяет текстуре равномерно рассеивать свет в разные направления, что особенно важно для отображения матовых и неровных поверхностей. Зеркальная составляющая создает блеск и отражения на более гладких участках поверхности, что позволяет подчеркнуть детали текстуры.

\section*{Вывод}

При процессе текстурирования с внесением изменений в нормали, главной целью является достижение высокой степени реалистичности.

Применяя текстурные карты, необходимо использовать объемные модели,
чтобы обеспечить правильное расположение текстуры на поверхности.

Выбор алгоритма обратной трассировки лучей обусловлен его способностью создавать более точное и реалистичное изображение поверхности, сохраняя детали и освещение. Отличным дополнением стала модель освещения Фонга, которая добавит еще больше реалистичности и позволит наиболее точно проанализировать наложение текстур.

Важно отметить, что благодаря такому подбору методов, при трассировке лучей мы сразу же учитываем отражающие и преломляющие способности поверхности, а также сразу можем накладывать текстурные карты. Это предоставляет возможность в несколько раз ускорить работу программы: нет необходимости производить вычисления поэтапно, анализируя одну и ту же поверхность несколько раз.
