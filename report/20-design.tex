\chapter{Конструкторская часть}

В данном разделе будут более подробно рассмотрены выбранные в предыдущем разделе методы и алгоритмы, а также предоставлены требования к программному обеспечению.

\section{Требования к программному обеспечению}
\begin{itemize}[label=\arabic*)]
	\item[-] Возможность отображать заданные объекты с учетом текстуры;
	\item[-] Поддержка различных типов источников света: точечные, направленные и рассеянные;
	\item[-] Учет теней, света и отражений;
	\item[-] Поддержка различных текстурных карт: карт высот, нормалей, параллактического отображения.
\end{itemize}

\section{Общий алгоритм решения задачи}
\begin{itemize}[label=\arabic*)]
	\item[-] Разместить источники света: определить их позиции и характеристики (типы и интенсивность);
	\item[-] Добавить объект на сцену;
	\item[-] Загрузить для объекта текстурные карты;
	\item[-] Трассировка лучей и визуализация:
	\begin{itemize}[label=\arabic*)]
		\item[-] Для каждой точки картинной плоскости сгенерировать луч, исходящий из источника наблюдения;
		\item[-] Проанализировать путь луча через сцену, учет взаимодействия с гранями объекта;
		\item[-] Внесение изменений в нормали объектов с учетом текстурных карт;
		\item[-] Расчет освещенности, теней и отражений на каждой точке поверхности объекта;
		\item[-] Формирование окончательного изображения на экране с учетом всех эффектов и модификаций.
	\end{itemize}
\end{itemize}

\section{Трассировка лучей}

В реальном мире свет исходит от источников освещения, отражается от
нескольких объектов и достигает наших глаз [4]. Процесс моделирования пути света называется трассировкой лучей. В компьютерной графике же используются алгоритмы обратной трассировки. Основная идея заключается в анализе путей лучей, исходящих не от источников света, а от “глаз”, смотрящих на предметы. Такой подход соответствует всем привычным нам законам физики, при этом является реальным для реализации, в отличие от симуляции фотонов.

\subsection{Свет}

Для наилучшего отображения реального освещения определим три типа
источников света:
\begin{itemize}[label=\arabic*)]
	\item[-] Точечные источники света: испускают его равномерно во всех направлениях из определенной точки в пространстве;
	\item[-] Направленные источники света: имеют фиксированное направление света, их можно рассматривать как бесконечно удаленные точечные источники, расположенные в заданном направлении;
	\item[-] Рассеянные источники света: привносят часть освещения в каждую точку сцены, независимо от ее положения. Упрощает визуализацию реальной модели, когда свет, достигнув объекта, рассеивает часть обратно в сцену.
\end{itemize}

Таким образом, на сцене мы определим один источник рассеянного света и несколько точечных и направленных источников.

\subsection{Наблюдатель и сцена}

Первое, что определяется на сцене – положение наблюдателя и окна просмотра. Введем обозначения. Пусть точка $O = (O_{x}, O_{y}, O_{z})$ – позиция камеры. Ширина и высота окна просмотра - $V_{w}, V_{h}$ соответственно. Так же определим расстояние от точки $O$ до плоскости окна – $d$, количество пикселей в окне приложения – $C_{w}$ (по ширине), $C_{h}$ (по высоте) и координаты пиксела на холсте - $C_{x}, C_{y}$.
Для перехода от координат холста ($C_{x}, C_{y}$) к пространственным координатам ($V_{x}, V_{y}, V_{z}$) необходимо будет выполнить следующие преобразования:
\begin{gather}
	V_{x} = C_{x}*\frac{V_{w}}{C_{w}};\hspace{1cm}V_{x} = C_{y}*\frac{V_{h}}{C_{h}};\hspace{1cm}V_{z} = d.
\end{gather}

Итак, для каждой точки холста мы можем определить соответствующую ему точку в окне просмотра.

\subsection{Лучи}

Когда мы определили источник лучей (т. $O$), и их направления ($V$), можно задать луч ($P$). Удобнее всего это сделать с помощью параметрического уравнения:
\begin{gather}
	P = O+t(V-O),
\end{gather}
где $t$ – любое неотрицательное вещественное число. Обозначим направление луча как: $\vec{D}=(V-O)$. Тогда:
\begin{gather}
	P = O+t\vec{D}.
\end{gather}

\subsection{Объекты}

Поскольку для представления объектов мы выбрали объемную модель, для универсальности программы можно использовать популярный формат – .obj (Wavefront OBJ) файлы. Данный формат – текстовый и используется для хранения трехмерных моделей. Он содержит следующую информацию:
\begin{itemize}[label=\arabic*)]
	\item[-] Вершины (англ. Vertices). Координаты точек в трехмерном пространстве, которые определяют форму объекта;
	\item[-] Нормали (англ. Normals). Направления поверхности в каждой вершине, важные для расчета освещения.
	\item[-] Текстурные координаты (англ. Texture Coordinates). Координаты, используемые для нанесения текстуры на поверхность модели.
	\item[-] Грани (англ. Faces). Определяют полигоны объекта, указывая индексы вершин, текстурных координат и нормалей, составляющих грани.
\end{itemize}

Таким образом, каждая грань объекта будет задана набором точек ($T_{i}$).

\subsection{Пересечение луча и грани}

Для того чтобы определить пересечения луча и грани можно проанализировать как проходит луч относительно каждого ребра при проходе их в заданном порядке. Если луч проходит с внешней стороны от какой-либо грани – пересечения не будет.

Для каждого ребра грани, образованного смежными вершинами $V_{i}$ и $V_{i+1}$, где $i = 0, 1, ..., n-1$ (где $n$ – количество вершин), мы должны выполнить следующие шаги:

\begin{enumerate}[label=\arabic*)]
	\item Пусть $P_{intersection}$ – точка пересечения луча с гранью. Чтобы ее определить с помощью параметрического уравнения необходимо вычислить параметр $t_{L}$. Для этого воспользуемся следующим уравнением:
	\begin{gather}
		t_{L} = \dfrac{N*P_L}{N*\vec{D}},
	\end{gather}
	где $P_L$ - вектор от начальной точки луча до первой точки вершины грани. Итак:
	\begin{gather}
		P_{intersection} = O+t_{L}\vec{D};
	\end{gather}
	\item Найдем вектора, соединяющие текущую вершину с точкой пересечения луча с плоскостью грани. Пусть $V_{i}$ и $V_{i+1}$ - текущее ребро грани.
	\item Найдем вектора ребер $E_{1}$ и $E_{2}$:
	\begin{gather}
		E_{1} = V_{i} - P_{intersection}; \hspace{1cm}
		E_{2} = V_{i+1} - P_{intersection};
	\end{gather}
	\item Вычислим векторное произведение ребер:
	\begin{gather}
		N_{edge} = E_{1} \times E_{2};
	\end{gather}
	\item Найдем скалярное произведение вектора ребра $N_{edge}$ и нормали грани $N$:
	\begin{gather}
		D_{P} = E_{1} \times E_{2};
	\end{gather}
	\item Если скалярное произведение меньше нуля, значит, луч проходит с внешней стороны от текущего ребра, и пересечения с гранью нет.
\end{enumerate}

\subsection{Диффузное отражение}

Диффузное отражение это процесс, при котором луч света, сталкиваясь с объектом, рассеивается обратно в сцену равномерно во всех направлениях. Именно оно придает матовым объектам матовость.
Рассмотрим луч света с направлением $\vec{L}$ и интенсивностью $I$, который достигает поверхности с нормалью $N$ . Представим интенсивность света как «ширину» луча. Его энергия распределяется по поверхности размером $A$. Отображу ситуацию на рисунке, обозначив вспомогательные углы и точки:

\begin{table}[H]
	\centering
	\begin{tabular}{p{1\linewidth}}
		\centering
		\includegraphics[height=0.4\linewidth]{include/2-1.png}
		\captionof{figure}{Диффузное отражение света}
		\label{img:2-1}
	\end{tabular}
\end{table}

Луч света шириной $I$ достигает поверхности в точке $T$ под углом $\beta$. Нормаль в $T$ это $\vec{N}$ , а переносимая лучом энергия распределяется по поверхности $A$. В случае, когда луч света имеет одинаковое направление с нормалью, $I=A$. С другой стороны, по мере того, как угол между $\vec{L}$ и $\vec{I}$ приближается к 90°, $A\to\infty$. Значит, $\lim_{x\to\infty} \frac{I}{A} = 0$. Выходит, для определения диффузного отражения, нам необходимо узнать что находится на промежутке, выяснив значение $\frac{I}{A}$.
Рассмотрим $RG$ – ширину луча. Так как эта прямая перпендикулярна лучу
света, $QRT = \alpha$. В треугольнике $QRT$ сторона $QR=\frac{1}{2}$, а $RT=\frac{A}{2}$. По определению $\cos{\alpha}=\frac{QR}{RT}=\frac{\frac{1}{2}}{\frac{A}{2}}=\frac{I}{A}$.
Так как $\alpha$ – угол между $\vec(N)$ и $\vec{L}$, мы можем использовать скалярное произведение этих векторов, чтобы выразить угол как:
\begin{gather}
	\frac{I}{A}=\cos{\alpha}=\frac{\left\langle{\vec{N}, \vec{L}}\right\rangle}{\left| {\vec{N}} \right| \left| {\vec{L}} \right|}.
\end{gather}
Мы получили простое уравнение, дающее нам отражаемую долю света в виде функции угла между нормалью поверхности и направлением света. Если у нас получилось, что значение $\alpha > 90°$, это означает, что свет освещает тыльную сторону поверхности, и учитывать это не нужно.
В итоге, мы можем записать уравнение диффузного отражения для вычисления общего количества света, получаемого точкой $T$ с нормалью $\vec{N}$ в сцене с
рассеянным светом с интенсивностью $I_{A}$ и $n$ точечных или направленных источников света с интенсивностью $I_{n}$, а так же световыми векторами $\vec{L_{n}}$:
\begin{gather}
	I_{T} =  I_{A} + \sum_{i=1}^n I_{i}*\frac{\left\langle{\vec{N}, \vec{L}}\right\rangle}{\left| {\vec{N}} \right| \left| {\vec{L}} \right|}.
\end{gather}

\subsection{Зеркальное отражение}

Зеркальное отражение отображает глянцевость объектов. Их вид меняется при смещении точки обзора. Рассмотрим луч света $\vec{L}$ . Нам нужно определить, сколько света от него отражается обратно в направлении точки обзора. Пусть $\vec{V}$ – вектор обзора, указывающий из точки $T$ в сторону камеры, а $\alpha$ – угол между $\vec{R}$ и $\vec{V}$ , тогда получим следующую картину:

\begin{table}[H]
	\centering
	\begin{tabular}{p{1\linewidth}}
		\centering
		\includegraphics[height=0.3\linewidth]{include/2-2.png}
		\captionof{figure}{Зеркальное отражение света}
		\label{img:2-2}
	\end{tabular}
\end{table}

При $\alpha = 0°$ весь свет отражается в направлении $\vec{V}$ . При $\alpha = 90°$ свет вообще не отражается. Как и в случае с диффузным отражением, нам нужно математическое выражение, позволяющее определить, что происходит при промежуточных значениях $\alpha$.
Данная модель не имеет физического прототипа, но отлично отражает не-
обходимые свойства и проста в вычислении.
Рассмотрим свойство $\cos{\alpha}: \cos{0} = 1 и \cos{\pm90} = 0$. Это мы и будем использовать. Теперь необходимо учесть глянцевость поверхности. Глянцевость – это мера того, как быстро уменьшается функция отражения при возрастании $\alpha$. Простой способ это сделать – возвести $\cos{\alpha}$ в положительную степень $s$:

\begin{table}[H]
	\centering
	\begin{tabular}{p{1\linewidth}}
		\centering
		\includegraphics[height=0.4\linewidth]{include/2-3.png}
		\captionof{figure}{График для $\cos{x}^{s}$}
		\label{img:2-3}
	\end{tabular}
\end{table}

$s$ – характеристика блеска поверхности. Для начала вычислим $\vec{R}$ из $\vec{N}$ и $\vec{L}$. Для этого разделим $\vec{L}$ на $\vec{L_{P}}$ и $\vec{L_{N}}$ так, чтобы $\vec{L} = \vec{L_{P}}+\vec{L_{N}}$:

\begin{table}[H]
	\centering
	\begin{tabular}{p{1\linewidth}}
		\centering
		\includegraphics[height=0.3\linewidth]{include/2-4.png}
		\captionof{figure}{Разделение $\vec{L}$ на $\vec{L_{P}}$ и $\vec{L_{N}}$}
		\label{img:2-4}
	\end{tabular}
\end{table}

$\vec{L_{N}}$ – проекция $\vec{L}$ на $\vec{N}$. Исходя из свойств скалярного произведения и того, что длина нормали равняется единице, длина проекции равна $\left\langle{\vec{N}, \vec{L}}\right\rangle$ Так как $\vec{L_{N}}||\vec{N}=\vec{N}*\left\langle{\vec{N}, \vec{L}}\right\rangle$. Получим $\vec{L_{P}}=\vec{L}-\vec{N}*\left\langle{\vec{N}, \vec{L}}\right\rangle$.
Теперь рассмотрим $\vec{R}$:

\begin{table}[H]
	\centering
	\begin{tabular}{p{1\linewidth}}
		\centering
		\includegraphics[height=0.3\linewidth]{include/2-5.png}
		\captionof{figure}{Расположение векторов $\vec{R}$ и $\vec{L}$}
		\label{img:2-5}
	\end{tabular}
\end{table}

Исходя из рисунка: $\vec{R} = \vec{L_{N}} -\vec{L_{P}} $. Подставив полученное выше:
\begin{gather}
	\vec{R} = \vec{N}*\left\langle{\vec{N}, \vec{L}}\right\rangle - \vec{L} + \vec{N}*\left\langle{\vec{N}, \vec{L}}\right\rangle = 2*\vec{N}*\left\langle{\vec{N}, \vec{L}}\right\rangle - \vec{L}
\end{gather}
Теперь, по аналогии с диффузным отражением, можно составить уравнение для зеркального:
\begin{gather}
	\vec{R} = 2*\vec{N}*\left\langle{\vec{N}, \vec{L}}\right\rangle - \vec{L};
\end{gather}
\begin{gather}
	I_{S} = I_{L}*\left( \frac{\left\langle{\vec{R}, \vec{V}}\right\rangle}{\left| {\vec{R}} \right| \left| {\vec{V}} \right|} \right)^{S}.
\end{gather}
Как и в случае с диффузным отражением, если $\cos{\alpha}$ оказался отрицательным, его нужно игнорировать. Кроме того, для матовых объектов выражение
зеркальности не должно вычисляться. Это можно предусмотреть заранее, пометив соответствующую поверхность, чтобы сократить количество вычислений.
Вычислив зеркальное и диффузное отражение, составим уравнение полного освещения в точке $T$:
\begin{gather}
	I_{T} = I_{A} + \sum_{i=1}^n I_{i}*\left[ {\frac{\left\langle{\vec{N}, \vec{L}}\right\rangle}{\left| {\vec{N}} \right| \left| {\vec{L}} \right|} + \left( \frac{\left\langle{\vec{R}, \vec{V}}\right\rangle}{\left| {\vec{R}} \right| \left| {\vec{V}} \right|} \right)^{S}} \right],
\end{gather}
где $I_{A}$ – интенсивность рассеянного света, $\vec{N}$ – нормаль к поверхности в
точке $T$, $V$ – вектор от $T$ к камере, $s$ – зеркальная характеристика поверхности, $I_{i}$ – интенсивность потока света $u$, $L_{i}$ – вектор из $T$ к свету $i$, а $R_{i}$ – вектор отражения в $T$ для потока света $i$.

\subsection{Тени}
Тени возникают, когда лучи света не могут достичь объекта из-за встреченного на пути препятствия.
Начнем с направленного света. Нам известна точка $T$ на поверхности, а также луч света $\vec{L}$ . Зная это, мы можем определить луч ($O+t\vec{L}$), проходящий из этой точки поверхности к бесконечно удаленному источнику света.
Если этот луч пересекается с чем-либо, то точка будет находится в тени и освещения от этого источника нужно игнорировать. В ином случае, мы добавляем освещение данного источника, как было показано ранее.
Точечные источники можно рассматривать так же, но надо учитывать то, что мы не хотим, чтобы объекты дальше источника света могли отбрасывать тени на $T$. Установим $t_{max} = 1$, чтобы при достижении источника света луч останавливался. Так же не стоит забывать, что $t_{min} = eps$, где $aps$ – очень близкое к нулю число справа, чтобы мы не нашли пересечение с поверхностью, из которой и пускаем луч.

\subsection{Отражения}

Когда мы смотрим в зеркало, мы видим отражаемые им лучи света. Они отражаются симметрично относительно нормали поверхности. Итак, нам необходимо выяснить, куда идет луч, отражаемый от зеркальной поверхности.
Таким образом, мы получим рекурсивный алгоритм. При его создании нам нужно убедится, что мы не порождаем бесконечный цикл. У нас будет два условия выхода: когда луч сталкивается с неотражающим объектом, и когда он ни с чем не сталкивается. Также важно вспомнить об эффекте «бесконечного коридора», который образуется, если поставить два зеркала друг напротив друга. Самый простой способ предотвратить бесконечный подсчет отражений: ограничить рекурсию каким-либо числом. В нашей задачи нет необходимости отображать зеркальные поверхности, а тем более визуализировать эффект «бесконечного коридора», поэтому можно обойтись достаточно малым значением в 2-3 единицы.

\section{Связь различных текстурных карт с данными о нормали}

Как было сказано ранее, существует несколько применяемых способов
вносить возмущения в нормали. Рассмотрим же соответствующие текстурные
карты от наиболее примитивной, к наиболее сложной.

\subsection{Карты высот (англ. height maps)}

Карты высот имеют черно-белый цвет, что означает, что все каналы RGB (red, green, blue) равны между собой. Таким образом, они хранят единственное значение. Пример:

\begin{table}[H]
	\centering
	\begin{tabular}{p{1\linewidth}}
		\centering
		\includegraphics[height=0.3\linewidth]{include/2-6.png}
		\captionof{figure}{Карта высот $h(x, y)$}
		\label{img:2-6}
	\end{tabular}
\end{table}

Чтобы получить возмущение, сначала нужно вычислить аппроксимацию производных по направлениям $x$ и $y$ с помощью разностного аналога [5]:
\begin{gather}
	h_{x}(x, y) = \frac{h(x+1, y) - h(x-1, y)}{2}; \hspace*{1cm} h_{y}(x, y) = \frac{h(x, y+1) - h(x, y-1)}{2}.
\end{gather}

В изображении это будет выглядеть так:
\begin{table}[H]
	\centering
	\begin{tabular}{p{1\linewidth}}
		\centering
		\includegraphics[height=0.3\linewidth]{include/2-7.png}
		\captionof{figure}{Аппроксимация по $x$ (слева) и $y$ (справа)}
		\label{img:2-7}
	\end{tabular}
\end{table}

Тогда ненормализованное значение нормали в текстеле $(x, y)$:
\begin{gather}
	N'(x, y)=(-h_{x}(x, y) - h_{y}(x, y), 1).
\end{gather}

\subsection{Карты нормалей (англ. normal maps)}
Карты нормалей содержат значения в двух каналах. Принято хранить $h_{x}$ в
канале R, а $h_{y}$ в G. Тогда значение канала B не используется, и все карты имеют голубоватый оттенок

В изображении это будет выглядеть так:
\begin{table}[H]
	\centering
	\begin{tabular}{p{1\linewidth}}
		\centering
		\includegraphics[height=0.3\linewidth]{include/2-8.png}
		\captionof{figure}{Карта нормалей}
		\label{img:2-8}
	\end{tabular}
\end{table}

\subsection{Карта параллактического отображения (англ. parallax map)}
С предыдущими типами карт есть проблема: при смещении угла обзора, это никак не влияет на рельеф. Если мы посмотрим вдоль реальной кирпичной стены под некоторым углом, мы не увидим зазор между кирпичами. Это происходит, потому что мы лишь изменяли нормали.
Идея параллакса заключается в том, что положение объектов относительно друг друга должно изменятся с движением наблюдателя. Неровности должны увеличиваться в высоту [5].

\begin{table}[H]
	\centering
	\begin{tabular}{p{1\linewidth}}
		\centering
		\includegraphics[height=0.3\linewidth]{include/2-9.png}
		\captionof{figure}{Фактическое положение поверхности определяется лучом взгляда (слева). Параллактическое отображение выполняет аппроксимацию, используя высоту для нахождения положения новой точки (справа).}
		\label{img:2-9}
	\end{tabular}
\end{table}

Теперь, помимо направления нормали, нам необходимо хранить высоту $h$. Учитывая местоположение с координатами текстуры $T$, высоту $h$, нормализованный вектор вида $V$ со значением высоты $V_{z}$ и горизонтальной составляющей $V_{xy}$, новая скорректированная по параллаксу текстурная координата:
\begin{gather}
	T_{adj} = T + \frac{h*V_{xy}}{V_{z}}.
\end{gather}
Однако, когда вектор обзора находится вблизи горизонта поверхности, небольшое изменение высоты приводит к большому смещению координат текстуры. Аппроксимация не будет корректной, поскольку полученное новое местоположение практически не коррелирует по высоте с исходным местоположением на поверхности. Чтобы решить эту проблему, мы может ограничить смещение: оно не должно превышать высоту. Тогда уравнение будет иметь вид:
\begin{gather}
	T_{adj} = T + h*V_{xy}.
\end{gather}
При крутых углах, это уравнение почти совпадает с исходным, поскольку $V_{z} \approx 1$. При малых углах смещение становится ограниченным. Но даже с такими ограничениями параллактическое смещение должно обеспечивать более реалистичное представление.

\section{Наложение текстуры на поверхность}

Теперь, когда у нас есть все необходимые уравнения и преобразования, остается лишь связать текстурную карту с поверхностью. То есть нужно определить, какой пиксел $h_{xy}$ текстурной карты соответствует точке на поверхности.

В .obj файле фигуры у нас заданы текстурные координаты каждой из вершин, а для того чтобы определить текстурные координаты ($u, v$) любой точки ($P$) грани выполним следующие шаги:

\begin{enumerate}[label=\arabic*)]
	\item Для каждого треугольника, образованного смежными вершинами $V_{i}, V_{i+1}, V{i+2}$ вычислим барицентрические координаты ($\alpha_{i}$, $\beta_{i}$, $\gamma{i}$). Для этого сначала определим векторы сторон треугольника:
	\begin{gather}
		\vec{AB} = B - A;
	\end{gather}
	\begin{gather}
		\vec{AC} = C - A.
	\end{gather}
	И вектор $\vec{AP}$:
	\begin{gather}
		\vec{AP} = P - A.
	\end{gather}
	Затем определим $\alpha_{i}$, $\beta_{i}$, $\gamma{i}$:
	\begin{gather}
		\alpha_{i} = \frac{\vec{AC}\times\vec{AP}}{\vec{AC}\times\vec{AB}};
	\end{gather}
	\begin{gather}
		\beta_{i} = \frac{\vec{AB}\times\vec{AP}}{\vec{AB}\times\vec{AC}};
	\end{gather}
	\begin{gather}
		\gamma_{i} = 1 - \alpha_{i} - \beta_{i}.
	\end{gather}
	\item Вычислим теперь текстурные координаты:
	\begin{gather}
		u = \sum_{i=1}^n \alpha_{i}u_{i};
		v = \sum_{i=1}^n \beta_{i}v_{i},
	\end{gather}
	где $\alpha_{i}, \beta_{i}$ - барицентрические координаты для $i$-й вершины, а $u_{i}$, $v_{i}$ - ее текстурные координаты.
\end{enumerate}

\section*{Вывод}

В данном разделе были подробно рассмотрены алгоритмы, необходимые для реализации поставленной цели.