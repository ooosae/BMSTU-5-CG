\chapter{Конструкторская часть}

В данном разделе будут более подробно рассмотрены выбранные в предыдущем разделе методы и алгоритмы, а также предоставлены требования к программному обеспечению.

\section{Требования к программному обеспечению. Общий алгоритм решения задачи}

Выделим следующие требования к ПО:
\begin{itemize}[label=\arabic*)]
	\item[-] возможность отображать заданные сферы с учетом текстуры;
	\item[-] поддержка различных типов источников света: точечные, направленные и рассеянные;
	\item[-] учет теней, света и отражений;
	\item[-] поддержка различных текстурных карт: высот, нормалей, параллактического отображения.
\end{itemize}

Общим алгоритм:
\begin{itemize}[label=\arabic*)]
	\item[-] Разместить источники света: определить их позиции и характеристики (типы и интенсивность);
	\item[-] Добавить сферы на сцену;
	\item[-] Загрузить для выбранных сфер текстурную карту;
	\item[-] Трассировка лучей и визуализация:
	\begin{itemize}[label=\arabic*)]
		\item[-] Для каждой точки картинной плоскости сгенерировать луч, исходящий из источника наблюдения;
		\item[-] Проанализировать путь луча через сцену, учесть взаимодействия с поверхностью ближайшей сферы;
		\item[-] Внести изменения в нормали сферы с учетом текстурных карт;
		\item[-] Рассчитать освещенность, тени и отражения на каждой точке поверхности сферы;
		\item[-] Сформировать окончательное изображение на экране с учетом всех эффектов и модификаций.
	\end{itemize}
\end{itemize}

\section{Трассировка лучей}

В реальном мире свет исходит от источников освещения, отражается от нескольких объектов и достигает глаз человека. Процесс моделирования пути света называется трассировкой лучей. В компьютерной графике же используются алгоритмы обратной трассировки. Основная идея заключается в анализе путей лучей, исходящих не от источников света, а от «глаз», смотрящих на предметы. Такой подход соответствует всем привычным нам законам физики, при этом является реальным для реализации, в отличие от симуляции фотонов \cite{b4}.

\subsection{Свет}

Для наилучшего отображения реального освещения определяют три типа источников света:
\begin{itemize}[label=\arabic*)]
	\item[-] Точечные источники света -- испускают его равномерно во всех направлениях из определенной точки в пространстве;
	\item[-] Направленные источники света -- имеют фиксированное направление света, их можно рассматривать как бесконечно удаленные точечные источники, расположенные в заданном направлении;
	\item[-] Рассеянные источники света -- привносят часть освещения в каждую точку сцены, независимо от ее положения. Упрощают визуализацию реальной модели, когда свет, достигнув объекта, рассеивает часть обратно в сцену.
\end{itemize}

Таким образом, на сцене определены несколько источников света: рассеянные, точечные и направленные.

\subsection{Наблюдатель и сцена}

Первое, что определяется на сцене -- положение наблюдателя и окна просмотра. Вводятся обозначения. Пусть точка $O = (O_{x}, O_{y}, O_{z})$ -- позиция камеры. Ширина и высота окна просмотра -- $V_{w}, V_{h}$ соответственно. Так же расстояние от точки $O$ до плоскости окна -- $d$, количество пикселей в окне приложения -- $C_{w}$ (по ширине), $C_{h}$ (по высоте) и координаты пиксела на холсте -- $C_{x}, C_{y}$.
Для перехода от координат холста ($C_{x}, C_{y}$) к пространственным координатам ($V_{x}, V_{y}, V_{z}$) необходимо выполнить следующие преобразования:
\begin{gather}
	V_{x} = C_{x}*\frac{V_{w}}{C_{w}};\hspace{1cm}V_{x} = C_{y}*\frac{V_{h}}{C_{h}};\hspace{1cm}V_{z} = d.
\end{gather}

Итак, для каждой точки холста можно определить соответствующую ему точку в окне просмотра.

\subsection{Лучи}

Когда определен источник лучей (т. $O$), и их направления ($V$), можно задать луч ($P$). Удобнее всего это сделать с помощью параметрического уравнения:
\begin{gather}
	P = O+t(V-O),
\end{gather}

где $t$ – любое неотрицательное вещественное число. Направление луча обозначено как: $\vec{D}=(V-O)$. Тогда:
\begin{gather}
	P = O+t\vec{D}.
\end{gather}

\subsection{Сферы}

Так как трассировка лучей – трудоемкий процесс, чаще всего вокруг всех объектов описываются сферы, и проверяется пересечение лучей сначала для них. Таким образом, можно исключить множество ненужных вычислений, и выиграть во времени выполнения программы. Так же, чтобы хранить полную информацию о сфере не нужно так много памяти. Даже не нужно определять сторону для текстуры, поскольку ею будет всегда лишь единственная видимая.

Определены имеющиеся данные о сфере: радиус $r$, и центр $S$. Теперь эти данные необходимо преобразовать, чтобы с ними было удобнее работать. Пусть точка на поверхности сферы -- $T$. По определению сферы:
\begin{gather}
	\rho(T, S) = r,
\end{gather}

где $\rho$ -- расстояние. Оно равно длине вектора:
\begin{gather}
	|T-S| = r.
\end{gather}

Длина вектора -- корень из его скалярного произведения с самим собой:
\begin{gather}
	\sqrt{\langle T-S, T-S \rangle} = r;
\end{gather}
\begin{gather}
	\langle T-S, T-S \rangle = r^2.
\end{gather}

\subsection{Пересечение луча и сферы}

Система из имеющихся выражений:
\begin{equation}
	\left\{
	\begin{array}{l}
		\langle T - S, T - S \rangle = r^2 \\
		P = O + t \vec{D}.
	\end{array}
	\right.
\end{equation}

Предположив, что луч и сфера пересекаются, и $T=P$, остается найти параметр $t$. Объединив имеющиеся уравнения из формулы (2.8) ($O-S=\vec{SO}$):
\begin{equation}
	\langle \vec{SO}+t\vec{D}, \vec{SO}+t\vec{D} \rangle = r^2.
\end{equation}

Далее использована дистрибутивность скалярного произведения:
\begin{equation}
	\langle \vec{SO}+t\vec{D}, \vec{SO} \rangle + \langle 	\vec{SO}+t\vec{D}, t\vec{D} \rangle = r^2;
\end{equation}
\begin{equation}
	\langle \vec{SO}, \vec{SO} \rangle + \langle t\vec{D}, t\vec{D} \rangle + \langle t\vec{D}, \vec{SO} \rangle + \langle \vec{SO}, t\vec{D} \rangle = r^2;
\end{equation}
\begin{equation}
	\langle t\vec{D}, t\vec{D} \rangle + 2\langle t\vec{D}, \vec{SO} \rangle + \langle \vec{SO}, \vec{SO} \rangle = r^2.
\end{equation}

$t$ вынесено за скалярное произведение:
\begin{equation}
	t^2\langle \vec{D}, \vec{D} \rangle + 2t\langle \vec{D}, \vec{SO} \rangle + \langle \vec{SO}, \vec{SO} \rangle - r^2 = 0.
\end{equation}

Получено квадратичное уравнение и его параметры:
\begin{equation}
	at^2+bt+c=0; \hspace*{1cm} \\
	\left\{
	\begin{array}{l}
		a=\langle \vec{D}, \vec{D} \rangle \\
		b=\langle \vec{D}, \vec{SO} \rangle \\
		c=\langle \vec{SO}, \vec{SO} \rangle - r^2.
	\end{array}
	\right.
\end{equation}

Решив его, можно найти пересечение луча со сферой. Определив пересечение луча со всеми сферами на сцене, можно получить наименьший параметр $t$, который будет соответствовать поверхности, расположенной как можно ближе к наблюдателю. Ее цвет и надо отобразить. В том случае, если луч не пересек ни одну сферу, пиксел должен быть закрашен цветом фона.

\subsection{Нормали сферы}

Для вычисления отражений необходимо уравнение нормали. Вектор нормали должен быть перпендикулярен поверхности и иметь длину 1, для любой точки сферы он лежит на прямой, проходящей через центр этой сферы:
\begin{equation}
	\vec{N}=\frac{T-S}{|T-S|}.
\end{equation}

\subsection{Диффузное отражение}

Диффузное отражение это процесс, при котором луч света, сталкиваясь с объектом, рассеивается обратно в сцену равномерно во всех направлениях. Именно оно придает матовым объектам матовость.
Рассмотрен луч света с направлением $\vec{L}$ и интенсивностью $I$, который достигает поверхности с нормалью $N$. Интенсивность света представлена как «ширина» луча. Его энергия распределяется по поверхности размером $A$. Ситуация отображена на рисунке, обозначены вспомогательные углы и точки:

\begin{table}[H]
	\centering
	\begin{tabular}{p{1\linewidth}}
		\centering
		\includegraphics[height=0.4\linewidth]{include/2-1.png}
		\captionof{figure}{Диффузное отражение света}
		\label{img:2-1}
	\end{tabular}
\end{table}

Луч света шириной $I$ достигает поверхности в точке $T$ под углом $\beta$. Нормаль в $T$ это $\vec{N}$ , а переносимая лучом энергия распределяется по поверхности $A$. В случае, когда луч света имеет одинаковое направление с нормалью, $I=A$. С другой стороны, по мере того, как угол между $\vec{L}$ и $\vec{I}$ приближается к 90°, $A\to\infty$. Значит,
\begin{equation}
\lim_{A\to\infty} \frac{I}{A} = 0.
\end{equation}

Выходит, для определения диффузного отражения, необходимо узнать что находится на промежутке, выяснив значение $\frac{I}{A}$. Рассматривалась $RG$ – ширина луча. Так как эта прямая перпендикулярна лучу света, $QRT = \alpha$. В треугольнике $QRT$ сторона $QR=\frac{1}{2}$, а $RT=\frac{A}{2}$. По определению:
\begin{equation}
\cos{\alpha}=\frac{QR}{RT}=\frac{\frac{1}{2}}{\frac{A}{2}}=\frac{I}{A}.
\end{equation}

Так как $\alpha$ – угол между $\vec{N}$ и $\vec{L}$, можно использовать скалярное произведение этих векторов, чтобы выразить угол:
\begin{gather}
	\frac{I}{A}=\cos{\alpha}=\frac{\left\langle{\vec{N}, \vec{L}}\right\rangle}{\left| {\vec{N}} \right| \left| {\vec{L}} \right|}.
\end{gather}

Получено простое уравнение, дающее отражаемую долю света в виде функции угла между нормалью поверхности и направлением света. Если получилось, что значение $\alpha > 90°$, это означает, что свет освещает тыльную сторону поверхности, и учитывать это не нужно.

В итоге, можно записать уравнение диффузного отражения для вычисления общего количества света, получаемого точкой $T$ с нормалью $\vec{N}$ в сцене с рассеянным светом с интенсивностью $I_{A}$ и $n$ точечных или направленных источников света с интенсивностью $I_{n}$, а так же световыми векторами $\vec{L_{n}}$:
\begin{gather}
	I_{T} =  I_{A} + \sum_{i=1}^n I_{i}*\frac{\left\langle{\vec{N}, \vec{L}}\right\rangle}{\left| {\vec{N}} \right| \left| {\vec{L}} \right|}.
\end{gather}

\subsection{Зеркальное отражение}

Зеркальное отражение отображает глянцевость объектов. Их вид меняется при смещении точки обзора. Рассмотрен луч света $\vec{L}$ . Необходимо определить, сколько света от него отражается обратно в направлении точки обзора. Пусть $\vec{V}$ – вектор обзора, указывающий из точки $T$ в сторону камеры, а $\alpha$ – угол между $\vec{R}$ и $\vec{V}$ , тогда:

\begin{table}[H]
	\centering
	\begin{tabular}{p{1\linewidth}}
		\centering
		\includegraphics[height=0.3\linewidth]{include/2-2.png}
		\captionof{figure}{Зеркальное отражение света}
		\label{img:2-2}
	\end{tabular}
\end{table}

При $\alpha = 0°$ весь свет отражается в направлении $\vec{V}$ . При $\alpha = 90°$ свет вообще не отражается. Как и в случае с диффузным отражением, необходимо получить математическое выражение, позволяющее определить, что происходит при промежуточных значениях $\alpha$.

Данная модель не имеет физического прототипа, но отлично отражает необходимые свойства и проста в вычислении.

Рассмотрено свойство $\cos{\alpha}: \cos{0} = 1 и \cos{\pm90} = 0$. Это будет использовано далее. Теперь необходимо учесть глянцевость поверхности. Глянцевость -- это мера того, как быстро уменьшается функция отражения при возрастании $\alpha$. Простой способ это сделать -- возвести $\cos{\alpha}$ в положительную степень $s$:

\begin{table}[H]
	\centering
	\begin{tabular}{p{1\linewidth}}
		\centering
		\includegraphics[height=0.4\linewidth]{include/2-3.png}
		\captionof{figure}{График для $\cos{x}^{s}$}
		\label{img:2-3}
	\end{tabular}
\end{table}

$s$ – характеристика блеска поверхности. Для начала вычислен $\vec{R}$ из $\vec{N}$ и $\vec{L}$. Для этого разделен $\vec{L}$ на $\vec{L_{P}}$ и $\vec{L_{N}}$ так, чтобы $\vec{L} = \vec{L_{P}}+\vec{L_{N}}$:

\begin{table}[H]
	\centering
	\begin{tabular}{p{1\linewidth}}
		\centering
		\includegraphics[height=0.3\linewidth]{include/2-4.png}
		\captionof{figure}{Разделение $\vec{L}$ на $\vec{L_{P}}$ и $\vec{L_{N}}$}
		\label{img:2-4}
	\end{tabular}
\end{table}

$\vec{L_{N}}$ – проекция $\vec{L}$ на $\vec{N}$. Исходя из свойств скалярного произведения и того, что длина нормали равняется единице, длина проекции равна $\left\langle{\vec{N}, \vec{L}}\right\rangle$ Так как $\vec{L_{N}}||\vec{N}=\vec{N}*\left\langle{\vec{N}, \vec{L}}\right\rangle$. Получается $\vec{L_{P}}=\vec{L}-\vec{N}*\left\langle{\vec{N}, \vec{L}}\right\rangle$. Рассмотрен $\vec{R}$:

\begin{table}[H]
	\centering
	\begin{tabular}{p{1\linewidth}}
		\centering
		\includegraphics[height=0.3\linewidth]{include/2-5.png}
		\captionof{figure}{Расположение векторов $\vec{R}$ и $\vec{L}$}
		\label{img:2-5}
	\end{tabular}
\end{table}

Исходя из рисунка: $\vec{R} = \vec{L_{N}} -\vec{L_{P}} $. Подставив полученное выше:
\begin{gather}
	\vec{R} = \vec{N}*\left\langle{\vec{N}, \vec{L}}\right\rangle - \vec{L} + \vec{N}*\left\langle{\vec{N}, \vec{L}}\right\rangle = 2*\vec{N}*\left\langle{\vec{N}, \vec{L}}\right\rangle - \vec{L}
\end{gather}

Теперь, по аналогии с диффузным отражением, можно составить уравнение для зеркального:
\begin{gather}
	\vec{R} = 2*\vec{N}*\left\langle{\vec{N}, \vec{L}}\right\rangle - \vec{L};
\end{gather}
\begin{gather}
	I_{S} = I_{L}*\left( \frac{\left\langle{\vec{R}, \vec{V}}\right\rangle}{\left| {\vec{R}} \right| \left| {\vec{V}} \right|} \right)^{S}.
\end{gather}

Как и в случае с диффузным отражением, если $\cos{\alpha}$ оказался отрицательным, его нужно игнорировать. Кроме того, для матовых объектов выражение зеркальности не должно вычисляться. Это можно предусмотреть заранее, пометив соответствующую поверхность, чтобы сократить количество вычислений.

Вычислив зеркальное и диффузное отражение, составлено уравнение полного освещения в точке $T$:
\begin{gather}
	I_{T} = I_{A} + \sum_{i=1}^n I_{i}*\left[ {\frac{\left\langle{\vec{N}, \vec{L}}\right\rangle}{\left| {\vec{N}} \right| \left| {\vec{L}} \right|} + \left( \frac{\left\langle{\vec{R}, \vec{V}}\right\rangle}{\left| {\vec{R}} \right| \left| {\vec{V}} \right|} \right)^{S}} \right],
\end{gather}

где $I_{A}$ – интенсивность рассеянного света, $\vec{N}$ -- нормаль к поверхности в точке $T$, $V$ -- вектор от $T$ к камере, $s$ -- зеркальная характеристика поверхности, $I_{i}$ -- интенсивность потока света $u$, $L_{i}$ -- вектор из $T$ к свету $i$, а $R_{i}$ -- вектор отражения в $T$ для потока света $i$.

\subsection{Тени}

Тени возникают, когда лучи света не могут достичь объекта из-за встреченного на пути препятствия.

Направленный света. Известна точка $T$ на поверхности, а также луч света $\vec{L}$ . Зная это, можно определить луч ($O+t\vec{L}$), проходящий из этой точки поверхности к бесконечно удаленному источнику света. Если этот луч пересекается с чем-либо, то точка будет находится в тени и освещения от этого источника нужно игнорировать. В ином случае, добавляется освещение данного источника, как было показано ранее.

Точечные источники можно рассматривать так же, но надо учитывать то, чтобы объекты дальше источника света не могут отбрасывать тени на $T$. Установлен $t_{max} = 1$, чтобы при достижении источника света луч останавливался. Так же не стоит забывать, что $t_{min} = eps$, где $eps$ – очень близкое к нулю число справа, чтобы пересечение с поверхностью, из которой и пускаем луч не учитывалось.

\subsection{Отражения}

Когда человек смотрит в зеркало, он видит отражаемые им лучи света. Они отражаются симметрично относительно нормали поверхности. Итак, необходимо выяснить, куда идет луч, отражаемый от зеркальной поверхности.

Таким образом, получен рекурсивный алгоритм. При его создании необходимо убедится, что не порождается бесконечный цикл. Задано два условия выхода: когда луч сталкивается с неотражающим объектом, и когда он ни с чем не сталкивается. Также важно вспомнить об эффекте «бесконечного коридора», который образуется, если поставить два зеркала друг напротив друга. Самый простой способ предотвратить бесконечный подсчет отражений: ограничить рекурсию каким-либо числом. В представленной задачи нет необходимости отображать зеркальные поверхности, а тем более визуализировать эффект «бесконечного коридора», поэтому можно обойтись достаточно малым значением в 2-3 единицы.

\section{Связь различных текстурных карт с данными о нормали}

Как было сказано ранее, существует несколько применяемых способов вносить возмущения в нормали. Рассмотрены соответствующие текстурные карты от наиболее примитивной, к наиболее сложной.

\subsection{Карты высот (англ. height maps)}

Карты высот имеют черно-белый цвет, что означает, что все каналы RGB (red, green, blue) равны между собой. Таким образом, они хранят единственное значение. Пример:

\begin{table}[H]
	\centering
	\begin{tabular}{p{1\linewidth}}
		\centering
		\includegraphics[height=0.4\linewidth]{include/2-6.png}
		\captionof{figure}{Карта высот $h(x, y)$}
		\label{img:2-6}
	\end{tabular}
\end{table}

Чтобы получить возмущение, сначала нужно вычислить аппроксимацию производных по направлениям $x$ и $y$ с помощью разностного аналога \cite{b5}:
\begin{gather}
	h_{x}(x, y) = \frac{h(x+1, y) - h(x-1, y)}{2}; \hspace*{1cm} h_{y}(x, y) = \frac{h(x, y+1) - h(x, y-1)}{2},
\end{gather}

где
\begin{itemize}
	\item $h(x, y)$ -- значение высоты в точке с координатами $(x, y)$;
	\item $h_{x}(x, y)$ -- аппроксимация производной высоты по направлению оси $x$ в точке $(x, y)$;
	\item $h_{y}(x, y)$ -- аппроксимация производной высоты по направлению оси $y$ в точке $(x, y)$.
\end{itemize}

В изображении это будет выглядеть так:
\begin{table}[H]
	\centering
	\begin{tabular}{p{1\linewidth}}
		\centering
		\includegraphics[height=0.4\linewidth]{include/2-7.png}
		\captionof{figure}{Аппроксимация по $x$ (слева) и $y$ (справа)}
		\label{img:2-7}
	\end{tabular}
\end{table}

Тогда ненормализованное значение нормали в текстеле $(x, y)$:
\begin{gather}
	N'(x, y)=(-h_{x}(x, y) - h_{y}(x, y), 1).
\end{gather}

\subsection{Карты нормалей (англ. normal maps)}
Карты нормалей содержат значения в двух каналах. Принято хранить $h_{x}$ в канале R, а $h_{y}$ в G. Тогда значение канала B не используется, и все карты имеют голубоватый оттенок.

В изображении это будет выглядеть так:
\begin{table}[H]
	\centering
	\begin{tabular}{p{1\linewidth}}
		\centering
		\includegraphics[height=0.4\linewidth]{include/2-8.png}
		\captionof{figure}{Карта нормалей}
		\label{img:2-8}
	\end{tabular}
\end{table}

\subsection{Карта параллактического отображения (англ. parallax map)}
С предыдущими типами карт есть проблема: при смещении угла обзора, это никак не влияет на рельеф. Если человек посмотрит вдоль реальной кирпичной стены под некоторым углом, он не увидит зазор между кирпичами. Это происходит, потому что были изменены лишь нормали.

Идея параллакса заключается в том, что положение объектов относительно друг друга должно изменятся с движением наблюдателя. Неровности должны увеличиваться в высоту \cite{b5}.

\begin{table}[H]
	\centering
	\begin{tabular}{p{1\linewidth}}
		\centering
		\includegraphics[width=0.9\linewidth]{include/2-9.png}
		\captionof{figure}{Фактическое положение поверхности определяется лучом взгляда (слева). Параллактическое отображение выполняет аппроксимацию, используя высоту для нахождения положения новой точки (справа).}
		\label{img:2-9}
	\end{tabular}
\end{table}

Теперь, помимо направления нормали, необходимо хранить высоту $h$. Учитывая местоположение с координатами текстуры $T$, высоту $h$, нормализованный вектор вида $V$ со значением высоты $V_{z}$ и горизонтальной составляющей $V_{xy}$, новая скорректированная по параллаксу текстурная координата:
\begin{gather}
	T_{adj} = T + \frac{h*V_{xy}}{V_{z}}.
\end{gather}

Однако, когда вектор обзора находится вблизи горизонта поверхности, небольшое изменение высоты приводит к большому смещению координат текстуры. Аппроксимация не будет корректной, поскольку полученное новое местоположение практически не коррелирует по высоте с исходным местоположением на поверхности. Чтобы решить эту проблему, можно ограничить смещение: оно не должно превышать высоту. Тогда уравнение будет иметь вид:
\begin{gather}
	T_{adj} = T + h*V_{xy}.
\end{gather}

При крутых углах, это уравнение почти совпадает с исходным, поскольку $V_{z} \approx 1$. При малых углах смещение становится ограниченным. Но даже с такими ограничениями параллактическое смещение должно обеспечивать более реалистичное представление.

\section{Наложение текстуры на поверхность}

Теперь, когда у представлены все необходимые уравнения и преобразования, остается лишь связать текстурную карту с поверхностью. То есть нужно определить, какой пиксел $h_{xy}$ текстурной карты соответствует точке на сфере. Определены координаты:
\begin{gather}
	x=u*M_w; \hspace*{1cm} y=v*M_h,
\end{gather}

где $M_w, M_h$ -- угол между положительным направлением оси $x$ и проекцией нормали к поверхности сферы на плоскость $XY$. Его можно вычислить по следующей формуле:
\begin{gather}
	\phi=\arctan{\frac{N_y}{N_x}},
\end{gather}

где $N_x$ -- проекция нормали на ось $x$, $N_y$ -- проекция нормали на ось $y$, $\phi$ -- угол, измеряющий поворот вокруг вертикальной оси $Z$ на плоскости $XY$. Он учитывает вращение окружности вокруг вертикальной оси, что является сферой. Переведя этот угол в диапазон $[0, 1]$ получается необходимое значение первой координаты точки на текстурной карте $u$:
\begin{gather}
	u=\frac{\phi+\pi}{2\pi}.
\end{gather}

Пусть $\theta$ -- угол между нормалью и положительным направлением оси $Z$:
\begin{gather}
	\theta=\arccos{N_z}.
\end{gather}

Данный угол определяет наклон нормали относительно вертикальной оси. На сфере он соответствует углу между нормалью и радиус-вектором точки на сфере. Он позволяет корректно распределить текстуру вдоль вертикальной оси сферы. Остается привести угол в необходимый диапазон, получив вторую координату точки на текстурной карте $v$:
\begin{gather}
	v=\frac{\theta}{\phi}.
\end{gather}

\section{Схема алгоритма трассировки луча}
\begin{table}[H]
	\centering
	\begin{tabular}{p{1\linewidth}}
		\centering
		\includegraphics[width=0.6\linewidth]{include/2-10.drawio.png}
		\captionof{figure}{Схема алгоритма трассировки луча}
		\label{img:2-10}
	\end{tabular}
\end{table}

\section*{Вывод}

В данном разделе были подробно рассмотрен алгоритм, необходимый для реализации поставленной цели, а так же построена его схема.