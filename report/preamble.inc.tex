\usepackage{cmap} % Улучшенный поиск русских слов в полученном pdf-файле
\usepackage[utf8]{inputenc} % Кодировка utf8
\usepackage[T2A]{fontenc} % Поддержка русских букв
\usepackage[english,russian]{babel} % Языки: русский, английский
%\usepackage{pscyr} % Нормальные шрифты
\usepackage{enumitem}
\usepackage{soul}
\usepackage{ulem}
\usepackage{float}
\usepackage{amssymb}
\usepackage[justification=centering]{caption}
\usepackage[group-separator={,}]{siunitx}
\usepackage{pdfpages}
\usepackage{amsmath}

% \usepackage[
% singlelinecheck=false % <-- important
% ]{caption}

% объявляем новую команду для переноса строки внутри ячейки таблицы
\newcommand{\specialcell}[2][c]{%
	\begin{tabular}[#1]{@{}c@{}}#2\end{tabular}}

\usepackage{caption}
\captionsetup{labelsep=endash}
\captionsetup[figure]{name={Рисунок}}

\usepackage{amsmath}

\usepackage{geometry}
\geometry{left=30mm}
\geometry{right=10mm}
\geometry{top=20mm}
\geometry{bottom=20mm}

\usepackage{titlesec}
\titleformat{\section}
{\normalsize\bfseries}
{\thesection}
{1em}{}
\titlespacing*{\chapter}{0pt}{-30pt}{8pt}
\titlespacing*{\section}{\parindent}{*4}{*4}
\titlespacing*{\subsection}{\parindent}{*4}{*4}

\usepackage{setspace}
\onehalfspacing % Полуторный интервал

\frenchspacing
\usepackage{indentfirst} % Красная строка

\usepackage{titlesec}
\titleformat{\chapter}{\LARGE\bfseries}{\thechapter}{20pt}{\LARGE\bfseries}
\titleformat{\section}{\Large\bfseries}{\thesection}{20pt}{\Large\bfseries}
% \titleformat{\chapter}[hang]{\Huge\bfseries}{\thechapter\hsp\textcolor{gray75}{|}\hsp}{0pt}{\Huge\bfseries}
\usepackage{listings}
\usepackage{xcolor}
\usepackage{color}

%\renewcommand{\thechapter}{\arabic{chapter}.}
%\renewcommand{\thesection}{\thechapter\arabic{section}.}
%\renewcommand{\thesubsection}{\thesection\arabic{subsection}.}
\renewcommand{\theequation}{\arabic{chapter}.\arabic{equation}}
\renewcommand\thefigure{\arabic{chapter}.\arabic{figure}}
\renewcommand\thetable{\arabic{chapter}.\arabic{table}}
\AtBeginDocument{
	\renewcommand\thelstlisting{\arabic{chapter}.\arabic{lstlisting}}
}

\lstset{ %
	language=C++,                 % выбор языка для подсветки (здесь это С++)
	basicstyle=\small\sffamily, % размер и начертание шрифта для подсветки кода
	numbers=left,               % где поставить нумерацию строк (слева\справа)
	numberstyle=\tiny,           % размер шрифта для номеров строк
	stepnumber=1,                   % размер шага между двумя номерами строк
	numbersep=5pt,                % как далеко отстоят номера строк от подсвечиваемого кода
	showspaces=false,            % показывать или нет пробелы специальными отступами
	showstringspaces=false,      % показывать или нет пробелы в строках
	showtabs=false,             % показывать или нет табуляцию в строках
	frame=single,              % рисовать рамку вокруг кода
	tabsize=2,                 % размер табуляции по умолчанию равен 2 пробелам
	captionpos=t,              % позиция заголовка вверху [t] или внизу [b] 
	breaklines=true,           % автоматически переносить строки (да\нет)
	breakatwhitespace=false, % переносить строки только если есть пробел
	escapeinside={\#*}{*)} ,  % если нужно добавить комментарии в коде
	keywordstyle=\color{purple},          % Стиль ключевых слов
	commentstyle=\color{dkgreen},       % Стиль комментариев
	stringstyle=\color{mauve}          % Стиль литералов
}


\usepackage{pgfplots}
\usetikzlibrary{datavisualization}
\usetikzlibrary{datavisualization.formats.functions}

\usepackage{graphicx}
\newcommand{\img}[3] {
	\begin{figure}[h!]
		\center{\includegraphics[height=#1]{assets/img/#2}}
		\caption{#3}
		\label{img:#2}
	\end{figure}
}
\newcommand{\boximg}[3] {
	\begin{figure}[h]
		\center{\fbox{\includegraphics[height=#1]{assets/img/#2}}}
		\caption{#3}
		\label{img:#2}
	\end{figure}
}

\usepackage[singlelinecheck=false]{caption} % Настройка подписей float объектов

\usepackage[unicode,pdftex]{hyperref} % Ссылки в pdf
\hypersetup{hidelinks}

\usepackage{csvsimple}

\makeatletter
\def\@biblabel#1{#1. }
\makeatother

\newcommand{\code}[1]{\texttt{#1}}

\addto\captionsrussian{\renewcommand{\bibname}{Список использованных источников}}
