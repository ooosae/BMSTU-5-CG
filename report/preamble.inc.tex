\usepackage{cmap} % Улучшенный поиск русских слов в полученном pdf-файле
\usepackage[utf8]{inputenc} % Кодировка utf8
\usepackage[T2A]{fontenc} % Поддержка русских букв
\usepackage[english,russian]{babel} % Языки: русский, английский
%\usepackage{pscyr} % Нормальные шрифты
\usepackage{enumitem}
\usepackage{soul}
\usepackage{ulem}
\usepackage{float}
\usepackage{amssymb}
\usepackage[justification=centering]{caption}
\usepackage[group-separator={,}]{siunitx}
\usepackage{pdfpages}
\usepackage{amsmath}
\usepackage{listings}
\usepackage{xcolor}

% \usepackage[
% singlelinecheck=false % <-- important
% ]{caption}

% объявляем новую команду для переноса строки внутри ячейки таблицы
\newcommand{\specialcell}[2][c]{%
	\begin{tabular}[#1]{@{}c@{}}#2\end{tabular}}

\usepackage{caption}
\captionsetup{labelsep=endash}
\captionsetup[figure]{name={Рисунок}}

\usepackage{amsmath}

\usepackage{geometry}
\geometry{left=30mm}
\geometry{right=10mm}
\geometry{top=20mm}
\geometry{bottom=20mm}

\usepackage{titlesec}
\titleformat{\section}
{\normalsize\bfseries}
{\thesection}
{1em}{}
\titlespacing*{\chapter}{0pt}{-30pt}{8pt}
\titlespacing*{\section}{\parindent}{*4}{*4}
\titlespacing*{\subsection}{\parindent}{*4}{*4}

\usepackage{setspace}
\onehalfspacing % Полуторный интервал

\frenchspacing
\usepackage{indentfirst} % Красная строка

\usepackage{titlesec}
\titleformat{\chapter}{\LARGE\bfseries}{\thechapter}{20pt}{\LARGE\bfseries}
\titleformat{\section}{\Large\bfseries}{\thesection}{20pt}{\Large\bfseries}
% \titleformat{\chapter}[hang]{\Huge\bfseries}{\thechapter\hsp\textcolor{gray75}{|}\hsp}{0pt}{\Huge\bfseries}
\usepackage{listings}
\usepackage{xcolor}
\usepackage{color}

%\renewcommand{\thechapter}{\arabic{chapter}.}
%\renewcommand{\thesection}{\thechapter\arabic{section}.}
%\renewcommand{\thesubsection}{\thesection\arabic{subsection}.}
\renewcommand{\theequation}{\arabic{chapter}.\arabic{equation}}
\renewcommand\thefigure{\arabic{chapter}.\arabic{figure}}
\renewcommand\thetable{\arabic{chapter}.\arabic{table}}
\AtBeginDocument{
	\renewcommand\thelstlisting{\arabic{chapter}.\arabic{lstlisting}}
}

\lstset{ %
	language=[Sharp]C,
	basicstyle=\ttfamily\footnotesize,
	keywordstyle=\color{blue}\bfseries,
	commentstyle=\color{gray}\itshape,
	stringstyle=\color{red},
	numberstyle=\tiny\color{gray},
	stepnumber=1,
	numbersep=10pt,
	showspaces=false,
	showstringspaces=false,
	showtabs=false,
	frame=single,
	tabsize=2,
	breaklines=true,
	breakatwhitespace=false,
	morekeywords={partial, var, yield, get, set, async, await}
}

\usepackage{pgfplots}
\usetikzlibrary{datavisualization}
\usetikzlibrary{datavisualization.formats.functions}

\usepackage{graphicx}
\newcommand{\img}[3] {
	\begin{figure}[h!]
		\center{\includegraphics[height=#1]{assets/img/#2}}
		\caption{#3}
		\label{img:#2}
	\end{figure}
}
\newcommand{\boximg}[3] {
	\begin{figure}[h]
		\center{\fbox{\includegraphics[height=#1]{assets/img/#2}}}
		\caption{#3}
		\label{img:#2}
	\end{figure}
}

\usepackage[singlelinecheck=false]{caption} % Настройка подписей float объектов

\usepackage[unicode,pdftex]{hyperref} % Ссылки в pdf
\hypersetup{hidelinks}

\usepackage{csvsimple}

\makeatletter
\def\@biblabel#1{#1. }
\makeatother

\newcommand{\code}[1]{\texttt{#1}}

\addto\captionsrussian{\renewcommand{\bibname}{Список использованных источников}}
